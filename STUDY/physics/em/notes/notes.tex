\documentclass[12pt]{article}

% page
\usepackage[margin=1in]{geometry}
\usepackage{microtype}
\usepackage[T1]{fontenc}
\usepackage{lmodern}
\usepackage{parskip}

% Math & Physics
\usepackage{amsmath, amssymb, mathtools}
\usepackage{physics}

% Units 
\usepackage{siunitx}
\sisetup{per-mode=symbol, detect-all, separate-uncertainty=true}

% Diagrams
\usepackage{tikz}
\usetikzlibrary{arrows.meta, positioning, calc}
\usepackage{circuitikz}

% Tables
\usepackage{booktabs}
\usepackage{tabularx}
\usepackage{longtable}
\usepackage{array}

% Figures / links
\usepackage{graphicx}
\usepackage{float}
\usepackage[hidelinks]{hyperref}

% Symbol System
% \sym{I} links to the table anchor "sym:I".
% Use \symsub{I}{avg}, \symsub{R}{eq}, etc. for common variants.
\newcommand{\sym}[1]{\hyperlink{sym:#1}{\ensuremath{#1}}}
\newcommand{\symsub}[2]{\hyperlink{sym:#1}{\ensuremath{#1_{\mathrm{#2}}}}}
\newcommand{\symsup}[2]{\hyperlink{sym:#1}{\ensuremath{#1^{#2}}}}

% Little helper
\newcommand{\key}[1]{\begin{center}\fbox{\parbox{0.94\linewidth}{#1}}\end{center}}
\newcolumntype{Y}{>{\raggedright\arraybackslash}X}

\title{AP Physics C: E\&M Notes}
\author{Miso Park}
\date{\today}

\begin{document}
\maketitle
\tableofcontents
\newpage

\section*{Notation / Symbol map}
% symbols-table.tex
% Symbol map (base symbols only). Variants go in Notes.

\begin{table}[H]
\centering
\begin{tabularx}{\linewidth}{@{}l l Y l Y@{}}
\toprule
\textbf{Name} & \textbf{Symbol} & \textbf{Meaning / definition} & \textbf{SI} & \textbf{Notes (variants)} \\
\midrule

Electric charge &
\hypertarget{sym:q}{$q$} &
Property that sources electric forces/fields &
\si{C} &
Often $q=ne$; common: $q_0$ \\

Current &
\hypertarget{sym:I}{$I$} &
Rate of flow of charge, $I=\dv{q}{t}$ &
\si{A} &
Common: $I_{\mathrm{avg}}, I_{\mathrm{rms}}, I_{\mathrm{max}}$ \\

Voltage &
\hypertarget{sym:V}{$V$} &
Electric potential difference (energy per unit charge) &
\si{V} &
Common: $V_0, V_{\mathrm{th}}$ \\

Resistance &
\hypertarget{sym:R}{$R$} &
Ratio $V/I$ for ohmic element &
\si{\ohm} &
$V=IR$; common: $R_{\mathrm{eq}}$ \\

\bottomrule
\end{tabularx}
\caption{Symbol map (base symbols only; variants listed in Notes).}
\end{table}


\section{CED Unit }
\subsection{OpenStax/CED}
\subsubsection{Kirchoff's Rule}
\key{\textbf{KCL:} sum of currents into a node = sum out.\quad
\textbf{KVL:} sum of potential changes around a loop = 0.}

\subsubsection{RC circuits}
\begin{align}
  \tau &= RC \\
  Q_{\text{charge}}(t) &= C\mathcal{E}\qty(1-e^{-t/RC}) \\
  Q_{\text{discharge}}(t) &= Q_0 e^{-t/RC}
\end{align}

\sub  section{Circuit sketch (starter)}
\begin{figure}[H]
\centering
\begin{circuitikz}
  \draw (0,0) to[battery1, l=$\mathcal{E}$] (0,2)
        to[R, l=$R$] (3,2)
        -- (3,0) -- (0,0);
\end{circuitikz}
\caption{Basic single-loop circuit.}
\end{figure}

\section{Mechanics}
\subsection{Work, energy, power}
\begin{align}
  W &= \int \vb{F}\cdot d\vb{r} \\
  K &= \tfrac12 mv^2 \\
  P &= \dv{W}{t} = \vb{F}\cdot\vb{v}
\end{align}

\end{document}