% Physics Notes Base Template (Mech + E&M)
% Save as: notes.tex
% Compile with: pdflatex / lualatex

\documentclass[11pt]{article}

% ---------- Page / typography ----------
\usepackage[margin=1in]{geometry}
\usepackage{microtype}
\usepackage[T1]{fontenc}
\usepackage{lmodern}
\usepackage{setspace}
\setstretch{1.08}
\usepackage{parskip}

% ---------- Math ----------
\usepackage{amsmath, amssymb, mathtools}
\usepackage{physics} % \dv, \pdv, \vb, \qty, \ket, \bra, \expval, etc.

% ---------- Units / constants ----------
\usepackage{siunitx}
\sisetup{
  per-mode=symbol,
  detect-all,
  separate-uncertainty=true,
  exponent-product=\cdot,
}

% ---------- Figures / diagrams ----------
\usepackage{graphicx}
\usepackage{float}
\usepackage{caption}
\usepackage{subcaption}

% TikZ for mechanics diagrams, FBDs, pulleys, etc.
\usepackage{tikz}
\usetikzlibrary{arrows.meta, positioning, calc}

% Circuits (built on TikZ)
\usepackage{circuitikz}

% ---------- Tables ----------
\usepackage{booktabs}
\usepackage{tabularx}
\usepackage{longtable}
\usepackage{array}

% ---------- Links ----------
\usepackage[hidelinks]{hyperref}

% ---------- Header/footer (optional) ----------
\usepackage{fancyhdr}
\pagestyle{fancy}
\fancyhf{}
\lhead{Physics Notes}
\rhead{\leftmark}
\cfoot{\thepage}

% ---------- Handy macros ----------
% Vectors: prefer bold vectors in physics notes
% (physics package already gives \vb{v} etc.)

% Common derivatives shortcuts (optional wrappers)
\newcommand{\dd}{\mathrm{d}}
\newcommand{\ee}{\mathrm{e}}

% Expectation / averages (physics already has \expval)
\newcommand{\avg}[1]{\qty\langle #1 \qty\rangle}

% Boxed key result
\newcommand{\key}[1]{\begin{center}\fbox{\parbox{0.92\linewidth}{#1}}\end{center}}

% ---------- Title ----------
\title{Physics C Notes}
\author{(Your Name)}
\date{\today}

\begin{document}
\maketitle
\tableofcontents
\newpage

% ==================================================
% SECTION TEMPLATE
% ==================================================
\section{Kinematics (example section)}

\subsection{Core definitions}
\begin{itemize}
  \item Position: $\vb{r}(t)$
  \item Velocity: $\vb{v}=\dv{\vb{r}}{t}$
  \item Acceleration: $\vb{a}=\dv{\vb{v}}{t}$
\end{itemize}

\subsection{Key equation / idea}
\key{If $\vb{a}$ is constant, then $\vb{v}=\vb{v}_0+\vb{a}t$ and $\vb{r}=\vb{r}_0+\vb{v}_0 t+\tfrac12 \vb{a}t^2$.}

\subsection{Worked example (skeleton)}
\textbf{Problem.} ...

\textbf{Setup.} ...

\textbf{Solve.} ...

\textbf{Answer.} ...

% ==================================================
% DIAGRAM TEMPLATE: free-body diagram
% ==================================================
\section{Diagram templates}

\subsection{Free-body diagram (block)}
\begin{figure}[H]
\centering
\begin{tikzpicture}[>=Latex]
  \draw[thick] (-1,0) -- (3,0);                 % ground
  \draw[thick] (0,0) rectangle (2,1);           % block
  \draw[->] (1,1) -- (1,2) node[above] {$N$};   % normal
  \draw[->] (1,0) -- (1,-1) node[below] {$mg$}; % weight
  \draw[->] (2,0.5) -- (3,0.5) node[right] {$F$}; % applied force
\end{tikzpicture}
\caption{Example free-body diagram.}
\end{figure}

\subsection{Circuit skeleton (battery + resistor)}
\begin{figure}[H]
\centering
\begin{circuitikz}
  \draw (0,0) to[battery1, l=$\mathcal{E}$] (0,2)
        to[R, l=$R$] (3,2)
        -- (3,0) -- (0,0);
\end{circuitikz}
\caption{Basic loop (useful for Kirchhoff practice).}
\end{figure}

% ==================================================
% TABLES go in a separate file (recommended)
% ==================================================
% See: symbols-table.tex (separate template)

\end{document}
