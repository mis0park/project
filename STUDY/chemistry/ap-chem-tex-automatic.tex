\documentclass[letterpaper, 11pt]{article}

% --- 패키지 로드 ---
\usepackage[utf8]{inputenc}
\usepackage[T1]{fontenc}
\usepackage{libertinus} 
\usepackage{lmodern, microtype}
\usepackage{amsmath, amssymb}
\usepackage{mhchem} 
\usepackage{xcolor}
\usepackage[most]{tcolorbox}
\usepackage{tikz}
\usepackage{todonotes}
\usepackage{enumitem}
\usepackage[hidelinks]{hyperref}
\usepackage{eso-pic} 

% [핵심 해결책] 이 패키지가 'Float(s) lost' 에러를 막아줍니다.
\usepackage{marginfix} 

% --- 페이지 레이아웃 설정 ---
\usepackage[
    left=0.75in, 
    top=1.875in, 
    bottom=0.75in, 
    textwidth=5.0in, 
    marginparsep=0.5in, 
    marginparwidth=1.6in
]{geometry}

% --- 변수 설정 ---
\newcommand{\currentunit}{Unit: Not Set}
\newcommand{\currenttopic}{Topic: Not Set}

% --- 레이아웃 그리기 ---
\AddToShipoutPictureBG{
    \begin{tikzpicture}[remember picture, overlay]
        % 수평선
        \draw[thick] ([shift={(0.75in, -1.625in)}]current page.north west) -- ([shift={(-0.75in, -1.625in)}]current page.north east);
        % 수직선
        \draw[thick] ([shift={(6in, -1.625in)}]current page.north west) -- ([shift={(6in, 0.5in)}]current page.south west);
        % Unit Label
        \node[anchor=north west, inner sep=0pt] at ([shift={(0.875in, -0.75in)}]current page.north west) {
            \fontsize{16pt}{16pt}\selectfont \currentunit
        };
        % Topic Label
        \node[anchor=north west, inner sep=0pt] at ([shift={(0.875in, -1.125in)}]current page.north west) {
            \fontsize{32pt}{32pt}\selectfont \textbf{\currenttopic}
        };
    \end{tikzpicture}
}

% --- 토픽 시작 커맨드 ---
\newcommand{\starttopic}[2]{
    \clearpage 
    \renewcommand{\currentunit}{#1}  
    \renewcommand{\currenttopic}{#2} 
    \phantomsection 
    \addcontentsline{toc}{section}{#2} 
}

% --- 본문 시작 ---
\begin{document}

% 1. 첫 번째 토픽
\starttopic{Unit 3: Cellular Energetics}{Topic 3.1 Enzyme Structure}

\section*{Main Notes}
Enzymes are biological catalysts.
\begin{itemize}
    \item Shape determines function.
    \item Active site specificity.
\end{itemize}

% 오른쪽 노트 예시
\marginpar{
    \section*{Keywords}
    Catalyst\\
    Substrate
}

\vspace{4in} % 일부러 공간을 채워서 페이지 끝으로 밀어봅니다.
Writing near the bottom of the page...
If I add a margin note here, it used to cause an error.

\marginpar{
    \textbf{Note:} This note is at the very bottom!
}

% 2. 두 번째 토픽 (이제 에러가 나지 않고 잘 보일 겁니다)
\starttopic{Unit 3: Cellular Energetics}{Topic 3.2 Cellular Respiration}

\section*{Glycolysis}
Now you can see Topic 3.2! The previous margin note has been handled safely.

\[ \ce{C6H12O6 + 6O2 -> 6CO2 + 6H2O} \]

\end{document}