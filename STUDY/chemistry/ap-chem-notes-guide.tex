\documentclass[letterpaper, 11pt]{article}

% ============================================================
% AP Chem Notes — CED-based TOC + Unit Preview pages (Letter)
% Layout: fixed header area + vertical margin-notes lane (right)
% ============================================================

% --- Packages ---
\usepackage[utf8]{inputenc}
\usepackage[T1]{fontenc}

% Pick ONE font family (Libertinus is great for “textbook-ish”)
\usepackage{libertinus}
\usepackage{microtype}

\usepackage{amsmath, amssymb}
\usepackage[version=4]{mhchem}

\usepackage{xcolor}
\usepackage[most]{tcolorbox}
\usepackage{tikz}
\usepackage{enumitem}
\usepackage[hidelinks]{hyperref}
\usepackage{eso-pic}

% Margin note stability (prevents “Float(s) lost”)
\usepackage{marginfix}

% --- Page layout (your Cornell-like lane) ---
\usepackage[
  left=0.75in,
  top=1.875in,
  bottom=0.75in,
  textwidth=5.0in,
  marginparsep=0.5in,
  marginparwidth=1.6in
]{geometry}

% --- Quiet common warning noise (helps avoid editor squiggle spam) ---
\hbadness=10000
\vbadness=10000
\hfuzz=9999pt
\vfuzz=9999pt
\setlength{\emergencystretch}{3em}
\tolerance=2000

% ============================================================
% Header labels (drawn on every page by TikZ overlay)
% ============================================================
\newcommand{\currentunit}{Unit: Not Set}
\newcommand{\currenttopic}{Topic: Not Set}

% ============================================================
% Page frame (line + divider + labels)
% ============================================================
\AddToShipoutPictureBG{
  \begin{tikzpicture}[remember picture, overlay]
    % horizontal rule
    \draw[thick] ([shift={(0.75in, -1.625in)}]current page.north west)
              -- ([shift={(-0.75in, -1.625in)}]current page.north east);

    % vertical divider (between main text and margin notes)
    \draw[thick] ([shift={(6in, -1.625in)}]current page.north west)
              -- ([shift={(6in, 0.5in)}]current page.south west);

    % Unit label
    \node[anchor=north west, inner sep=0pt] at ([shift={(0.875in, -0.75in)}]current page.north west) {%
      \fontsize{16pt}{16pt}\selectfont \currentunit
    };

    % Topic label
    \node[anchor=north west, inner sep=0pt] at ([shift={(0.875in, -1.125in)}]current page.north west) {%
      \fontsize{32pt}{32pt}\selectfont \textbf{\currenttopic}
    };
  \end{tikzpicture}
}

% ============================================================
% Hyperlink helpers (manual TOC without needing \tableofcontents)
% ============================================================
\newcommand{\unitanchor}[1]{\hypertarget{#1}{}}
\newcommand{\unitlink}[2]{\hyperlink{#1}{#2}}

% ============================================================
% Start pages
% ============================================================
\newcommand{\startfront}[2]{%
  \clearpage
  \renewcommand{\currentunit}{#1}%
  \renewcommand{\currenttopic}{#2}%
}

\newcommand{\startunitpreview}[2]{%
  \clearpage
  \unitanchor{#2}%
  \renewcommand{\currentunit}{#1}%
  \renewcommand{\currenttopic}{Unit Preview}%
}

\newcommand{\starttopic}[2]{%
  \clearpage
  \renewcommand{\currentunit}{#1}%
  \renewcommand{\currenttopic}{#2}%
}

% ============================================================
% Right-lane note box (margin notes)
% ============================================================
\newcommand{\rnote}[1]{%
  \marginpar{%
    \footnotesize
    \begin{tcolorbox}[
      colback=yellow!10,
      colframe=black!60,
      boxrule=0.6pt,
      sharp corners,
      left=5pt,right=5pt,top=5pt,bottom=5pt
    ]
    #1
    \end{tcolorbox}
  }%
}

% ============================================================
% Unit Preview block styles
% ============================================================
\newtcolorbox{previewbox}[1]{%
  colback=white,
  colframe=black!70,
  boxrule=0.8pt,
  sharp corners,
  left=10pt,right=10pt,top=9pt,bottom=9pt,
  title=\textbf{#1}
}

\newcommand{\diagramslist}[1]{%
  \begin{previewbox}{Must-draw diagrams (your “visual anchors”)}
  \begin{itemize}[leftmargin=1.2em, itemsep=2pt]
    #1
  \end{itemize}
  \end{previewbox}
}

\newcommand{\topicslist}[1]{%
  \begin{previewbox}{CED Topic checklist}
  \begin{itemize}[leftmargin=1.2em, itemsep=2pt]
    #1
  \end{itemize}
  \end{previewbox}
}

\newcommand{\skillslist}[1]{%
  \begin{previewbox}{What you should be able to do}
  \begin{itemize}[leftmargin=1.2em, itemsep=2pt]
    #1
  \end{itemize}
  \end{previewbox}
}

% ============================================================
% Manual “CED Table of Contents” page
% ============================================================
\newcommand{\cedtoc}{%
  \startfront{AP Chemistry}{CED Contents}

  \rnote{\textbf{How to use:}\\
  Click unit titles to jump to the Unit Preview page.\\[4pt]
  Keep Unit Previews short; dump details into Topic pages.}

  \begin{previewbox}{AP Chem CED Units (clickable)}
  \begin{enumerate}[leftmargin=1.4em, itemsep=4pt]
    \item \unitlink{U1}{Unit 1: Atomic Structure and Properties}
    \item \unitlink{U2}{Unit 2: Molecular and Ionic Compound Structure and Properties}
    \item \unitlink{U3}{Unit 3: Intermolecular Forces and Properties}
    \item \unitlink{U4}{Unit 4: Chemical Reactions}
    \item \unitlink{U5}{Unit 5: Kinetics}
    \item \unitlink{U6}{Unit 6: Thermodynamics}
    \item \unitlink{U7}{Unit 7: Equilibrium}
    \item \unitlink{U8}{Unit 8: Acids and Bases}
    \item \unitlink{U9}{Unit 9: Applications of Thermodynamics}
  \end{enumerate}
  \end{previewbox}

  \begin{previewbox}{Note}
  If you have a local file named \texttt{ap-chem-ced.md}, keep it beside this \texttt{.tex} as your “source of truth”.
  This template is pre-filled with the standard AP Chem unit/topic outline so it compiles cleanly even without that file.
  \end{previewbox}
}

% ============================================================
% Document
% ============================================================
\begin{document}

% ---------- CED Table of Contents ----------
\cedtoc

% ---------- Unit Preview Pages (one per unit) ----------

% Unit 1
\startunitpreview{Unit 1: Atomic Structure and Properties}{U1}
\rnote{\textbf{Core move:}\\
Count moles, analyze composition, read PES, predict trends.}
\topicslist{
  \item Moles + molar mass; mass composition
  \item Empirical / molecular formula basics
  \item Atomic structure; electron configuration
  \item Photoelectron spectroscopy (PES) interpretation
  \item Periodic trends (radius, IE, EN) \& reasoning
}
\diagramslist{
  \item PES spectrum sketch \& how peaks map to subshells
  \item Periodic trend arrows + “why” notes
}
\skillslist{
  \item Convert between mass, moles, particles
  \item Use composition data to infer formulas
  \item Justify trends with effective nuclear charge/shielding
}

% Unit 2
\startunitpreview{Unit 2: Molecular and Ionic Compound Structure and Properties}{U2}
\rnote{\textbf{Core move:}\\
Structure \(\rightarrow\) bonding model \(\rightarrow\) properties.}
\topicslist{
  \item Types of chemical bonds; electronegativity \& polarity
  \item Lewis diagrams, resonance, formal charge
  \item VSEPR shapes + molecular polarity
  \item Hybridization (conceptual) \& bond angles
  \item Ionic solids, metallic bonding (property links)
  \item Molecular orbital (qualitative: bond order, magnetism)
}
\diagramslist{
  \item Lewis + resonance set (NO\(_3^-\), CO\(_3^{2-}\), SO\(_4^{2-}\))
  \item VSEPR shape map (AX\(_m\)E\(_n\)) + polarity check
  \item MO diagrams for O\(_2\), N\(_2\) (qualitative)
}
\skillslist{
  \item Draw Lewis reliably; choose best resonance contributor
  \item Predict shape + polarity; count electron domains correctly
  \item Connect bonding type to melting point/conductivity trends
}

% Unit 3
\startunitpreview{Unit 3: Intermolecular Forces and Properties}{U3}
\rnote{\textbf{Core move:}\\
IMFs explain phase + properties.}
\topicslist{
  \item Intermolecular forces (LDF, dipole, H-bonding, ion-dipole)
  \item Properties: boiling/melting, viscosity, surface tension
  \item Solids/liquids/gases; particulate diagrams
  \item Ideal gas law, kinetic molecular theory
  \item Solutions + solubility (qualitative patterns)
}
\diagramslist{
  \item IMF ranking ladder + “why” examples
  \item Heating curve (phase change) + where energy goes
  \item PV/nRT sketch + KMT particle diagram
}
\skillslist{
  \item Rank substances by BP/VP using structure + IMFs
  \item Explain deviations from ideal behavior conceptually
}

% Unit 4
\startunitpreview{Unit 4: Chemical Reactions}{U4}
\rnote{\textbf{Core move:}\\
Balance + classify + quantify.}
\topicslist{
  \item Net ionic equations; spectators
  \item Precipitation \& solubility rules (patterns)
  \item Oxidation numbers; redox basics
  \item Stoichiometry in reactions; limiting reagent
  \item Titrations (basic setup + endpoints concept)
}
\diagramslist{
  \item Redox “electron bookkeeping” template
  \item Titration curve idea (strong/strong baseline)
}
\skillslist{
  \item Write net ionic equations
  \item Identify oxidized/reduced; connect to electron transfer
  \item Solve limiting reagent / yield problems cleanly
}

% Unit 5
\startunitpreview{Unit 5: Kinetics}{U5}
\rnote{\textbf{Core move:}\\
Rates reveal mechanisms.}
\topicslist{
  \item Rate laws; determining order from data
  \item Integrated rate laws (conceptual + basic calculations)
  \item Arrhenius; activation energy
  \item Mechanisms; rate-determining step; intermediates
  \item Catalysis (how it changes pathway)
}
\diagramslist{
  \item Energy diagram: catalyzed vs uncatalyzed
  \item Mechanism step map (slow step \(\rightarrow\) rate law pieces)
}
\skillslist{
  \item Extract rate law from experiments
  \item Interpret energy diagrams + explain catalyst effect
  \item Match plausible mechanism to rate law (qualitative)
}

% Unit 6
\startunitpreview{Unit 6: Thermodynamics}{U6}
\rnote{\textbf{Core move:}\\
Energy accounting + directionality.}
\topicslist{
  \item Enthalpy; calorimetry; Hess’s law
  \item Entropy (qualitative drivers)
  \item Gibbs free energy; spontaneity; \(\Delta G\)
  \item Thermodynamic favorability vs rate (contrast)
}
\diagramslist{
  \item Enthalpy diagram (exo/endo) + sign conventions
  \item \(\Delta G = \Delta H - T\Delta S\) sign table
}
\skillslist{
  \item Use Hess’s law / formation enthalpies
  \item Reason about spontaneity using \(\Delta G\), \(\Delta S\)
}

% Unit 7
\startunitpreview{Unit 7: Equilibrium}{U7}
\rnote{\textbf{Core move:}\\
Q vs K + Le Châtelier.}
\topicslist{
  \item Equilibrium expressions; \(K\) and \(Q\)
  \item Manipulating \(K\); linking to \(\Delta G^\circ\) (qualitative)
  \item Le Châtelier’s principle (stress-response reasoning)
  \item Solubility equilibrium \(K_{sp}\) (conceptual + basic)
}
\diagramslist{
  \item ICE-table skeleton (generic)
  \item \(Q\) vs \(K\) direction decision flow
}
\skillslist{
  \item Write \(K\) expressions correctly
  \item Predict shifts and composition changes
}

% Unit 8
\startunitpreview{Unit 8: Acids and Bases}{U8}
\rnote{\textbf{Core move:}\\
pH + buffers + titration logic.}
\topicslist{
  \item Acid/base definitions; strong vs weak
  \item pH/pOH; \(K_a, K_b\); relationships
  \item Buffers; Henderson–Hasselbalch (when appropriate)
  \item Titrations (weak/strong curve features)
}
\diagramslist{
  \item pH scale + log intuition
  \item Titration curve feature map (equivalence, buffer region)
}
\skillslist{
  \item Compute/interprete pH in common scenarios
  \item Explain buffer behavior qualitatively + with \(K_a\)
}

% Unit 9
\startunitpreview{Unit 9: Applications of Thermodynamics}{U9}
\rnote{\textbf{Core move:}\\
Electrochem + \(\Delta G\) + equilibrium connection.}
\topicslist{
  \item Galvanic vs electrolytic cells
  \item Cell diagrams; half-reactions
  \item \(E^\circ\), \(\Delta G^\circ\), and \(K\) relationships
  \item Nernst equation (conceptual + basic use)
}
\diagramslist{
  \item Cell schematic (anode/cathode, electron flow)
  \item \(E\)–\(Q\) trend via Nernst (qualitative)
}
\skillslist{
  \item Identify anode/cathode; direction of electron flow
  \item Connect \(E^\circ\) sign to spontaneity
}

% ---------- Topic pages (optional placeholders) ----------
% Keep these as templates; add real notes later.
% \starttopic{Unit 2: Molecular and Ionic Compound Structure and Properties}{Topic 2.7 VSEPR}
% \rnote{\textbf{Key:} domains, shape, polarity.}
% % ... your notes ...

\end{document}
