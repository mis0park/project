
\documentclass[12pt]{article}
\usepackage[margin=1in]{geometry}
\usepackage{amsmath,amssymb}
\usepackage{graphicx}
\usepackage{booktabs}
\usepackage{hyperref}
\usepackage{setspace}
\onehalfspacing

\title{AP Biology Unit 3 Study Guide \\ Energetics}
\author{}
\date{}

\begin{document}
\maketitle

\section{Unit Overview}
This unit focuses on how biological systems acquire, transform, and use energy.
The core logic is that energy flows downhill, gradients form, and those gradients
are used to do work.

\section{Enzymes}

\subsection{What is an Enzyme?}
An enzyme is a protein catalyst that increases the rate of a chemical reaction by
lowering the activation energy barrier without being consumed or changing the
overall Gibbs free energy of the reaction.

\subsection{Gibbs Free Energy Diagrams}
Gibbs free energy diagrams show energetic feasibility, not reaction speed.
The vertical axis represents Gibbs free energy ($G$), and the horizontal axis
represents reaction progress.

\begin{itemize}
  \item The height of the peak represents activation energy.
  \item The difference between reactants and products represents $\Delta G$.
  \item Enzymes lower activation energy but do not change $\Delta G$.
\end{itemize}

\subsection{Mechanism of Enzyme Action}
Enzymes lower activation energy by orienting substrates, straining bonds,
stabilizing the transition state, and creating favorable microenvironments.
The induced-fit model explains how active sites change shape upon substrate binding.

\section{Environmental Effects on Enzymes}

\subsection{Denaturation}
Denaturation is the loss of a protein's three-dimensional structure due to
disruption of hydrogen bonds, ionic interactions, or disulfide bridges, leading
to loss of enzyme function.

\subsection{Temperature and pH}
Temperature affects molecular motion, while pH alters amino acid charge states.
Both can change enzyme shape and reduce activity if conditions move outside the
optimal range.

\section{Cellular Energy}

\subsection{Energy in Biological Systems}
Energy is the capacity to do work. Living systems maintain internal order by
exporting entropy and therefore require constant energy input.

\subsection{ATP and Energy Coupling}
ATP couples exergonic and endergonic reactions. Hydrolysis of ATP releases energy
that drives nonspontaneous cellular processes.

\section{Photosynthesis}

\subsection{Conceptual Overview}
Photosynthesis converts light energy into chemical energy via electron flow and
proton gradients.

\subsection{Light Reactions}
Light reactions occur in the thylakoid membrane. Water is split at Photosystem II,
electrons move through an electron transport chain, protons are pumped into the
thylakoid lumen, and ATP synthase produces ATP.

\subsection{Calvin Cycle}
The Calvin cycle occurs in the stroma and uses ATP and NADPH to reduce carbon
dioxide into sugar precursors. RuBisCO catalyzes carbon fixation.

\section{Cellular Respiration}

\subsection{Conceptual Overview}
Cellular respiration extracts energy from organic molecules to generate ATP
through electron transport and chemiosmosis.

\subsection{Electron Transport Chain}
Electrons donated by NADH and FADH$_2$ move through the electron transport chain,
driving proton pumping across the inner mitochondrial membrane. Oxygen is the
final electron acceptor.

\subsection{Fermentation}
Fermentation regenerates NAD$^+$ when oxygen is absent, allowing glycolysis to
continue.

\section{Unifying Principle}
Electron flow creates proton gradients, and those gradients are used to do work.
This logic is conserved across photosynthesis and cellular respiration.

\end{document}
